% !TEX program = xelatex

\documentclass[letterpaper,12pt]{article}
\usepackage{tabularx} % extra features for tabular environment
\usepackage{amsmath}  % improve math presentation
\usepackage{graphicx} % takes care of graphic including machinery
\usepackage{enumitem}
\usepackage{verbatim}
\usepackage{listings}
\usepackage{xcolor}
\usepackage{fontawesome5}
\usepackage[margin=1in,letterpaper]{geometry} % decreases margins
\usepackage{cite} % takes care of citations
\usepackage{xeCJK}
\usepackage{fontspec}
\usepackage{setspace}
\usepackage{indentfirst}
\usepackage[final]{hyperref} % adds hyper links inside the generated pdf file
\hypersetup{
	colorlinks=true,       % false: boxed links; true: colored links
	linkcolor=blue,        % color of internal links
	citecolor=blue,        % color of links to bibliography
	filecolor=magenta,     % color of file links
	urlcolor=blue
}
\usepackage{blindtext}
\usepackage{datetime}
\renewcommand{\today}{\number\year 年 \number\month 月 \number\day 日}

\newcommand\link[1]{\href{#1}{\color{black}\faLink}}
%++++++++++++++++++++++++++++++++++++++++

\setCJKmainfont[
    RawFeature     = +fwid
]{Source Han Serif CN}
\setCJKsansfont[
    RawFeature     = +fwid
]{Sarasa Gothic SC}
\setCJKmonofont[
    UprightFont    = * Normal,
    BoldFont       = * Bold,
    AutoFakeSlant  = 0.1763,
    Scale          = 0.8903,
    RawFeature     = +fwid
]{Source Han Sans CN}
\setmainfont[
    Extension      = .otf,
    UprightFont    = *-Regular,
    BoldFont       = *-Bold,
    ItalicFont     = *-Italic,
    BoldItalicFont = *-BoldItalic,
    Scale          = 1.1
]{LibertinusSerif}
\setmonofont[
    UprightFont    = * Light,
    BoldFont       = * Semibold,
]{Iosevka Slab}

\setlength{\parindent}{2em}
\begin{document}
\onehalfspacing
\title{项目总结}
\author{岳泽龙}
\date{\today}
\maketitle

\section{项目信息}

\subsection{项目名称}
    为 openEuler 添加 Xfce 桌面环境并能够运行在树莓派 4B 上

\subsection{方案描述}
    在 openEuler 原有包的基础上,打包添加 Xfce 桌面运行所需的依赖,使 Xfce 可以正常运行,并打包其所需的配套软件,使桌面环
    境达到基本可用的状态。在 Raspberry Pi 4B 上测试安装后,将桌面环境制作进入 Raspberry Pi 4B 的镜像中,测试直接刷入镜像能
    否使用桌面环境,并进行相关稳定性测试。

\subsection{时间规划}
基本按照原时间计划,完成了该项目的目标。

\subsubsection{实际流程}
\begin{itemize}
    \item 第一阶段(第一周到第五周):打出可以正常使用,但配套软件并不齐全的 Xfce 桌面环境。
    \item 第二阶段(第六周到第八周):打出软件齐全的 Xfce 桌面环境,并对全新安装进行测试,编写了一部分文档。
    \item 第三阶段(第九周到第十三周):制作了带 Xfce 桌面环境的镜像,整理了部分脚本和软件包,向上游做出一定贡献。
\end{itemize}

\subsubsection{原定安排}
\begin{table}[!ht]
    \setlength\extrarowheight{2pt}
    \begin{tabularx}{\textwidth}{|c|X|}
    \hline
    日期        & \multicolumn{1}{c|}{安排} \\
    \hline
    7.1 - 7.7   & 通过大量浏览 openEuler 现有包的 spec,学习 RPM 打包规范,并尝试给 openEuler 打一些源里暂时没有的包进行练手                                      \\ \hline
    7.8 - 7.14  & 根据 openEuler 的实际需求,首先查看 X11 相关的工具以及软件包是否齐全,如果 Xfwm 和 Xfce 有相关依赖不在软件源中,则进行补全,尝试在本周内能打出可运行的 Xfwm             \\ \hline
    7.15 - 7.21 & 在本周内尝试打出 Xfce,并添加默认的字体配置文件,如果所需的中文 / 英文 / 等宽字体在软件源中不存在,则进行补全。并检测登录 / 注销 / 重启 / 关机等功能能否正常运行                 \\ \hline
    7.28 - 8.3  & 在本周内为已有的 Xfce 桌面环境进行一部分功能的添加,例如网络功能,检查 NetworkManager 的状态,并打出 Xfce 的网络管理器,蓝牙管理器等等,并进行蓝牙以及网络图形界面管理功能的验证     \\ \hline
    \textbf{第一阶段}    & \textbf{第一阶段的主要任务是学习 RPM 打包规范,补全 Xfce 依赖的相关软件包,并打出一个具有基本功能,可以正常使用的 Xfce 桌面环境} \\ \hline
    8.4 - 8.10  & 在本周内打完大部分的 Xfce 配套软件包,诸如终端模拟器,文件管理器,浏览器等等,并逐一测试功能,对于 Xfce 团队没有额外开发,但是工作 / 生活中会用到的软件,进行挑选,打包,并在将来撰写的文档中进行推荐 \\ \hline
    8.11 - 8.17 & 完成中期检查报告,对项目完成度进行评估,将没有完成的部分进行整理,并根据实际进度进行后续的工作安排,对前一段的工作进行总结,不足的地方进行反思                                    \\ \hline
    8.18 - 8.24 & 将还未打出的包进行补充,尽量在本周内完成所有相关包的打包,并对所有包进行记录整理,功能测试,等等                                                           \\ \hline
    8.25 - 8.31 & 将已有的系统清空,在干净的存储卡上全新安装,记录安装过程,编写文档。将安装过程中遗漏的依赖,软件包等记录下来。对安装的系统再次进行功能测试                                      \\ \hline
    \textbf{第二阶段}    & \textbf{第二阶段的主要任务是将 Xfce 相关所有的软件包打完,并撰写一部分文档,验证可重复安装性,为结束该任务做准备} \\ \hline
    9.1 - 9.7   & 对前一周记录的依赖,软件包进行添加,完善文档,并按照新打出的包重新进行全新安装,如有问题,则进一步解决                                                        \\ \hline
    9.8 - 9.14  & 将桌面环境以及 base 环境以镜像的形式进行打包,测试镜像,修复镜像,并根据实际情况,清理非必要包,对镜像体积进行精简                                               \\ \hline
    9.15 - 9.21 & 按照之前的记录,完善文档,介绍 Xfce 的使用,推荐一部分非必要软件包。进行最后的测试和总结,修复能发现的漏洞,梳理提交                                              \\ \hline
    9.22 - 9.30 & 对本次活动进行总结,撰写总结报告,按照导师和项目组的要求完成评价,总结等等,并对自己整个阶段的表现和收获进行整理,发布在博客上,完成整个活动的参与阶段                                \\ \hline
    \textbf{第三阶段}    & \textbf{第三阶段的主要任务是撰写完文档,将安装过程及使用过程中可能遇到的问题尽可能找到,并进行修复,尽力提高项目的完成度,结束项目} \\ \hline
    \end{tabularx}
\end{table}

\newpage

\section{项目总结}

\subsection{项目成果}
    \begin{itemize}
        \item 可以正常运行的 Xfce 桌面及其配套软件、插件等
            \begin{itemize}
                \item 对于网络设置,连接无线网络,可以使用 \verb!network-manager-applet!
                \item 对于蓝牙传输文件,可以使用 \verb!blueman!
                \item 对于文件管理器,可以使用 Xfce 配套的 \verb!Thunar!
                \item 对于浏览器,可以使用 Xfce 配套的 \verb!Midori! 或者 \verb!Firefox!(建议)
            \end{itemize}
        \item 编写了打包 Xfce 桌面及相关配套软件的脚本 \link{https://github.com/dragonjacson/summer2020_openeuler/blob/master/scripts/buildall.sh}
        \item 向上游提交了 Pull Request:
            \begin{itemize}
                \item 给 \verb!openeuler/raspberrypi! 的构建脚本补充构建桌面的功能 \link{https://gitee.com/openeuler/raspberrypi/pulls/23}
                \item 向 \verb!src-openeuler/xfce4-panel! 的 \verb!SPEC! 中添加宏及相关判断,使其可以正常打包 \link{https://gitee.com/src-openeuler/xfdesktop/pulls/2}
                \item 对 \verb!src-openeuler/xfdesktop! 的 \verb!SPEC! 中的日期进行修正 \link{https://gitee.com/src-openeuler/xfce4-panel/pulls/2}
                \item 向 \verb!src-openeuler/lightdm-gtk! 提交源码及 SPEC 文件 \link{https://gitee.com/src-openeuler/lightdm-gtk/pulls/1}
                \item 向 \verb!openeuler/community! 中提交添加包的申请 \link{https://gitee.com/openeuler/community/pulls/1133}
                \begin{itemize}
                    \item \verb!libgxim! \link{https://gitee.com/raspi-oo/libgxim}
                    \item \verb!imsettings! \link{https://gitee.com/raspi-oo/imsettings}
                    \item \verb!fcitx! \link{https://gitee.com/raspi-oo/fcitx}
                    \item \verb!fcitx-qt5! \link{https://gitee.com/raspi-oo/fcitx-qt5}
                    \item \verb!fcitx-libpinyin! \link{https://gitee.com/raspi-oo/fcitx-libpinyin}
                \end{itemize}
                \item 向 \verb!dylanaraps/neofetch! 中添加 openEuler 的 Logo(已合并) \link{https://github.com/dylanaraps/neofetch/pull/1510}
            \end{itemize}
        \item 向上游提交了 Issue:
            \begin{itemize}
                \item \verb!src-openeuler/polkit! 相关,可能导致桌面用户无法直接关机 / 重启等的 Issue \link{https://gitee.com/src-openeuler/polkit/issues/I1Q8EF?from=project-issue}
            \end{itemize}
    \end{itemize}

\subsection{遇到的问题及解决方案}
    \begin{itemize}
        \item 对于 \verb!src-openeuler! 中已经存在的包,部分包无法完成打包(例如 \verb!xfce4-panel!),查询了 RPM 打包文档
        以及源码的 \verb!Makefile!、\verb!CMakeList! 等文件后,找到了需要关闭的选项,并在 \verb!SPEC! 中添加宏及相关判断,
        使软件包可以正常打包。在此过程中,也了解到大概需要去哪里查找软件包的可选项,并对 RPM SPEC 的规范更加了解。
        \item 在编写打包全部桌面相关文件脚本的时候,在解决依赖的问题上,面临几个选择,首先是可以使用 \verb!mock!,它会自动
        解决依赖顺序,另外是可以使用 \verb!Haskell! 编写的脚本来解决依赖顺序的问题。但是,前者需要一个稳定可用的源,并且配
        置上比较麻烦,后者的话依赖过多的其他 \verb!Haskell! 的包,想让整个程序运行起来比较麻烦。考虑到相互依赖,且在源里本
        身不存在的包并不算太多,可以手动解决依赖,并将正确的编译顺序使用 list 来记住,在编写打包脚本时,按照 list 里的顺序
        进行打包,实践之后是可以成功的,并且由于包不多,维护起来也不算麻烦。
    \end{itemize}

\section{总结与思考}

\subsection{收获}
    \begin{itemize}
        \item 在完成项目的期间,熟悉了 RPM SPEC 规范,以后可以向上游社区做出更多贡献
        \item 在完成项目的期间,编写了很多脚本,在此之前编写脚本的次数较少,经常是「笨拙」地使用补全或者 \verb!zsh! 的
        相关功能完成一部分工作,但是这样费时费力,并且很多时候不够方便,也不能让别人简单地复现你的操作。完成这次项目的
        过程中,编写了许多脚本,感受到了使用脚本的便利,以后在使用 Linux 的时候会考虑编写更多的脚本。
        \item 在完成项目的期间,导师会在每一阶段给出合适的意见,让我对每一阶段的任务都有更清晰的认识,不会因为无所事事
        或者没有明确的目标而耽误不少时间。同时,对于需要提交给社区的代码,导师也给予了相关的建议,根据建议修改后的代码
        更加符合社区需要的规范。可以说整个项目的完成离不开导师的指导。在与导师沟通的过程中,我也增加了与社区合作工作的
        经验,让我了解到一个颇具规模的开源社区是如何运转的。
    \end{itemize}

\subsection{不足}
    \begin{itemize}
        \item 由于编写脚本的经验相对不足,许多脚本不太符合应有的规范或者「最佳实践」,大多数脚本仅仅满足「能用」,甚至有
        些 \verb!Hack!,以后还需要学习更多相关知识。
        \item 中期遇到的 \verb!Midori! 浏览器打开网页会崩溃的问题,初期以为是 \verb!WebKit2GTK3! 版本过旧导致的问题,但事
        实上,使用了 openEuler 20.09 中较新的 \verb!WebKit2GTK3! 仍会遇到该问题(但可以打开部分网页,例如学校用于网络认证
        的 Portal),在尝试了安装字体、更改 \verb!locale!,更改编译选项,更改设置后,仍无法正常使用 \verb!Midori!,使用其
        他系统的 \verb!Midori! 却可以正常浏览网页。对此,可能需要使用 \verb!gdb! 或者其他方式来进行分析原因,但由于能力不
        够,暂时无法修复,只能使用 \verb!Firefox! 来进行网页浏览。
    \end{itemize}

\end{document}
